\documentclass{article} \usepackage{hyperref} \usepackage{graphicx}

\begin{document}

\title{Paper Review - Zipf's Law} \author{Anastasios Antoniadis}

\maketitle

\begin{abstract}

Whenever phenomena with apparent irregularity are observed over large
samples of data, it is important to detect the statistical properties of the
system which cause such regularities. This paper presents Zipf's Law, a law
proposed by G.K. Zipf to model the regularity behind some naturally
occurring phenomena. This is, in essence, an algebraically decaying function
describing the probability distribution.
\end{abstract}

\section{Introduction}

Zipf's Law is a law first formulated by G. K. Zipf using mathematical
statistics. This paper presents on various formulations of Zipf's law and
describes a few attempts at statistically explaining its theoretical
underpinnings. Zipf's Law models an empirical phenomenon which has been
observed in data studied in several physical and social sciences and the
author attempts to consolidate various cases in which Zipf's law has been
empirically shown to hold and concludes with the investigation of real data
distributions, as an independent verification of the law.

\section{Formulation of Zipf's Law}
\subsection{Simple form}

The simple form of Zipf's suggests that $rxr = constant$ where $r$ is the
rank of $xr$ is the size of the $r$th data value in an ordered set. This
rank-size relation is known as Zipf's Law and its graph is a rectangular
hyperbola.

\subsection{Generalized form}

A main drawback of the Zipf's Law is that the phenomena observed by Zipf and
justified by statistical rationale lead to a family of distributions
described by the zeta function.

\section{Theoretical foundation of Zipf's Law}
\subsection{Cumulative Advantage Distribution}

Price presented the cumulative advantage distribution, which can be derived
as a stochastic birth process.

\subsection{Mandelbrot's derivation}

By assuming that the aim of language is to transmit the most information per
symbol with the least effort Mandelbrot obtained the following relationship:

\begin{equation}
  \label{simple_equation} f(r) = K(r+c)^{-\theta}
\end{equation}

where $f(r)$ is the word frequency and $r$ is the rank of the word. The
constant $c$ improves the fit for small $r$ and the exponent improves the
fit for large $r$.

\subsection{Simon's approach}

Simon expanded on Zipf's work by describing a set of empirically derived
skew distribution functions.

\subsection{Rationale behind Zipf's law}

The empowerment of Zipf's law assumes some properties about the system being
studied. In the case of a system limited to the usage frequency of words in
literature, Simon observed that the stochastic process by which words are
chosen to be included in written text follows two steps:

\begin{itemize}
  \item By process of association, i.e., sampling earlier segments of
    his/her word sequences.
  \item By imitation, i.e., sampling from other works by self or other
    authors.
\end{itemize}

The assumptions made in Simon's formula are:
\begin{enumerate}
   \item The probability that the $(T + 1)$st word has appeared exactly $r$
     times is proportional to the total number of occurrences of all words
     that have appeared r times.
   \item For large $T$, there is a constant probability $ω$ that the $(T +
     1)$st word has not appeared in the first $T$ words. item
\end{enumerate}

The process of association produces words which can only be the results of
the first assumption, while the process of imitation can also produce words
which are the results of assumption 2.

\section{Verification of Zipf's law on real distributions}

The author attempted to verify Zipf's law using real life data, in
particular using a database of statistics of some NBA players for the years
1991-92. Many of the statistics demonstrated roughly hyperbolic graphs,
which lead to the claim that Zipf's law was empirically verified for this
particular real life database.

\section{Paper Evaluation}
\subsection{Strengths}
\begin{itemize}
\item Some categories of phenomena display some apparent regularities but
  are not easy to explain using simple laws of nature. However, it is
  essential to be able to model such categories of phenomena even if it is
  not (yet) possible to fully model establish a mathematical which reasons
  them. Having some information about a set of data can make a lot of
  difference compared to having no information at all. This is why Zipf's
  law has so many applications in spite of its empirical nature which is a
  limiting factor to its application.
\end{itemize}
\subsection{Weaknesses}
\begin{itemize}
\item As already mentioned the empirical nature of Zipf's law is itself
  limiting. A series of unbiased experiments is required in order to
  ascertain that the behaviour described by Zipf's law is present. Moreover,
  while in some cases it is possible to explain the behaviour observed (for
  instance in the case of Big cities vs Small cities) in other cases it is
  possible that the behaviour observed is accidental or the subset of a
  behaviour described by a more generalized mathematical model.
\item The verification method used in Section 5: ``Verification of Zipf's
  law on real distributions'' is an anti-paradigm of how scientific
  evaluation experiments should be demonstrated in scientific work. The
  selective presentation of a plot and the statement that the expected
  behaviour was observed in other statistics is by no means a scientifically
  approved method of verification\footnote{While I am aware of the nature of
    this paper which was an introductory article regarding Zipf's law, its
    generality and its derivations and the properties of systems obeying
    Zipf's law rather than a full work intended for peer review, for the
    purpose of this homework I consider my my argument valid. In fact this
    paper reminded of an article by Jeffrey D. Ullman on his website:
    \href{http://infolab.stanford.edu/~ullman/pub/experiments.pdf}{Experiments
      as Research Validation – Have We Gone too Far?}}.
\end{itemize}


\end{document}
