\documentclass{article} \usepackage{hyperref} \usepackage{graphicx}

\begin{document}

\title{Paper Review - Constructing Call Graphs of Scala Programs}
\author{Anastasios Antoniadis}

\maketitle

\section{Introduction}

This paper revises existing call graph construction algorithms ---
Name-Based Resolution and Rapid Type Analysis --- to Scala features
such as traits and abstract type members in order to demonstrate that
careful handling of Scala characteristics greatly improves call graph
precision.

Moreover, the paper explains that the source-based algorithms
developed by the authors are much more precise than bytecode-based
algorithms as a result of the significant amount of type information
lost due to the transformations and optimizations performed by the
Scala compiler and the generation of code which uses reflection which
is hard to analyse.

Traits are a cornerstone feature of Scala aiming to distribute the
functionality of an object over multiple reusable components. Method
call behavior in Scala is based on the linearization of a class. The
presence of traits complicates the construction of call graphs since
method calls that occur in a trait typically cannot be resolved by
consulting the class hierarchy alone. In order to achieve a precise
and efficient solution it is necessary to determine the set of
combinations of traits that actually occur in a program.

Furthermore, regarding abstract type members it is necessary to
understand how abstract type members are bound to concrete types as a
result of trait composition in order to resolve method calls. The
approach followed by the authors considers how abstract type members
are bound to concrete types in observed combinations of traits.

Closures constitute another Scala feature which allows functions to be
bound to variables and passed as arguments to other functions. In this
case the analysis in not performed at source-level, it is instead
performed after the Scala compiler has desugared the code.

Calls on the this reference are common in both Java and Scala. In some
cases it is possible to resolve such class more precisely by
exploiting the knowledge that the caller and the callee must be
members of the same object. Special care is taken in the case of super
calls where the signatures of methods which are targets of all
reachable super calls are blacklisted.

Experimental evaluation on a collection of Scala programs showed that
bytecode-analysis is much more imprecise compared to source-code
analysis. Name-based analysis lead to a very significant precision
loss compared to subtype-based analysis. In the case of call sites on
receivers with abstract types the algorithm based on the set of
concrete types that may be bound to abstract type members is more
precise compared to the bound-based algorithm. The special handling of
this calls did not lead to significant precision improvement. The
running time of the analyses improves as the precision increases due
to the lower amount of call graph edges generated. The algorithms
focused on abstract type members and special handling of this calls
have similar execution times to the one handling traits since such
calls are a relatively small fraction of all call sites in the
benchmark programs. Finally, a comparison between the name-based
resolution algorithm and the most precise algorithm which handles all
the previously mentioned Scala features shows that a very significant
amount of the calls which are identified as polymorphic by the
name-based analysis are resolved to monomorphic in the latter case.

\section{Paper Evaluation}
\subsection{Strengths}
\begin{itemize}
\item This is the first piece work done on call graph construction in
Scala. It serves as a good starting point mainly because it explains
the behaviour of some cornerstone Scala features and adapts basic call
graph construction algorithms to these features, despite the fact that
there is no real innovation in the paper. At the same time the authors
identify the limitations of bytecode-bases call graph construction
algorithms and their origins.
\item The algorithms were evaluated using a reasonably wide variety of
Scala programs with different characteristics. Moreover, the
evaluation did not solely focus on precision, the authors also
evaluated the running time of each algorithm and at the same time
identified the weaknesses of the algorithms developed such as the
inefficacy of this handling in many cases and the reason behind why
such inefficacy is observed.
\end{itemize}

\subsection{Weaknesses}
\begin{itemize}
\item My understanding is that the algorithms evaluated in this paper
do not rely on points-to information. However, call graph creation and
points-to analysis are mutually recursive problems. Considering that
the main measurement in the evaluation section is the precision of
algorithms I would expect the implementation and evaluation of
algorithms which rely on points-to information and some observations
on how Scala features affect the precision of such algorithms.
\end{itemize}
\end{document}
